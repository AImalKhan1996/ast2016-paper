%!TEX root=../ast2016.tex

\begin{abstract}
%% Chris's original bullet points:
%   \begin{itemize}
%       \item Relational schemas can include constraints that contain business logic so need to be tested.
%       \item Mutation analysis can be used to model faults of omission and commission in constraints.
%       \item As with mutation analysis elsewhere, it can be costly (regardless of prior researched optimisations).
%       \item Virtual analysis exploits an existing model of database behaviour, used elsewhere for data generation, to reduce this \todo{to/by substantial figure here}.
%       \item Therefore we can mutate even large schemas quickly, making it a practically feasible testing technique.
%   \end{itemize}

%% Why this topic is important
Relational databases are a vital component of many modern software applications.
%
%% What the schema is, introduce integrity constraints
Key to the definition of the database schema---which specifies what types of data will be stored in the database and the structure in which the data is to be organized---are integrity constraints.
%
%% What integrity constraints are, and why we need to test them
Integrity constraints are conditions that protect and preserve the consistency and validity of data in the database, preventing data values that violate their rules from being admitted into database tables. They encode logic about the application concerned, and like any other component of a software application, need to be properly tested.
%
%% Introduce mutation analysis
Mutation analysis is a technique that has been successfully applied to integrity constraint testing, seeding database schema faults of both omission and commission. Yet, as for traditional mutation analysis for program testing, it is costly to perform,
%% cut because we're not mentioning scalability now...
% and tends not to scale well where large numbers of mutant schemas are involved.
since the test suite under analysis needs to be run against each individual mutant to establish whether or not it exposes the fault.
%% The crux of our technique
One overhead incurred by database schema mutation is the cost of communicating with the database management system (DBMS). In this paper we seek to eliminate this cost by performing mutation analysis {\it virtually} on a local model of the DBMS, rather than on an actual, running instance hosting a real database. We present an empirical evaluation of our virtual technique that shows that significantly improves the time taken to perform mutation analysis, improving the scalability of the technique.
%
%% Some stats
% TODO

\end{abstract}
