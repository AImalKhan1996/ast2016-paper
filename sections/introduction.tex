%!TEX root=../ast2016.tex

\section{Introduction}
\label{sec:introduction}

% GMK NOTE: Here are some data points about the prevalence of queries of StackExchange about SQL databases

% select
% Count(Id)
% from Posts
% where
% (
% Tags like '%sql%'
% and Tags not like '%nosql%'
% or Tags like '%mysql%'
% or Tags like '%oracle%'
% or Tags like '%postgres%'
% or Tags like '%sqlite%'
% or Tags like '%derby%'
% )

% TOTAL = 876022 posts

% I have confirmed that screening out NoSQL using the above technique does work correctly and avoids the over-counting
% that would occur if the not like was not included

% GMK NOTE: Here are some data points about the prevalence of questions on StackExchange about NoSQL databases

\begin{itemize}

  \item Relational databases form a key part of many different applications, finding use in a broad range of domains from highly popular internet browser applications (Chrome ref, Firefox ref) to making important contributions to political campaigns (Amazon ref), by providing a reliable way to store and retrieve data.

  \item Information is stored according to a relational schema, expressing the structure to be used as a series of tables and columns, usually expressed with the structured query language (SQL). To ensure the data is valid and can be retrieved efficiently a series of constraints can also be defined. These express restrictions such as that particular columns must be unique for each row in a table or that rows in one table must correctly refer to rows in another.

  \item As these constraints prevent invalid or incomplete data from being stored in the database, they are essential to ensuring that applications using this data behave correctly, as well as that any information derived from the data is not incorrect. Therefore, it is vital that these constraints are tested.

  \item To this end, we have previously proposed a technique to exercise the constraints of a relational database schema, using data generated by a search-based technique and a range of constraint coverage criteria, designed to identify different types of programmer fault.

  \item In order to evaluate how well this data is able to detect faults, and provide a means of comparing this to other data generation approaches for schema testing in the future, a mutation analysis technique was developed. This systematically injects faults into a schema, one at a time, to produce mutant schemas and then executes the test data against each, identifying whether the modification leads to a change in behaviour, `killing' the mutant.

  \item However, as with mutation analysis of other software artefacts, such as programs, the computational cost of this can become very high, as the number of mutants increases rapidly as the size of the mutated artefact increases. Prior work to ameliorate this for mutation analysis of relational database schemas sought to incorporate techniques inspired by those applied to mutation analysis of programs (Mutation2013 ref). In contrast, this paper proposes a novel technique that makes use of an existing model of relational databases to avoid the need for a database during testing, leading to significant efficiency improvements whilst maintaining the same effectiveness (\todo{Perhaps include a phrase here to indicate it works across multiple DBMSs -- it isn't just that we're choosing a slow DBMS to beat, we're beating three of the most popular in performance.}).

  \item \todo{Structure of the paper comments here}

\end{itemize}
