%!TEX root=../ast2016.tex

% Important numbers
\newcommand{\numschemas}{30\xspace}

% Highlight outstanding issues
\newcommand{\todo}[1]{{\color{red}$<$TODO: #1$>$}}
\newcommand{\query}[1]{{\color{blue}$<$QUERY: #1$>$}}
\newcommand{\answer}[1]{{\color{ForestGreen}$<$ANSWER: #1$>$}}
\newcommand{\thesis}[1]{{\color{Plum}$<$THESIS PROSE: #1$>$}}
\newcommand{\paper}[2]{{\color{Orange}$<$#1: #2$>$}}

% Definition-type environments
\newtheorem{definition}{Definition}
\newtheorem{criterion}{Criterion}

% Extra heading levels
\newcommand{\inlineheading}[1]{\vspace{1ex}{\noindent}{\bf #1.}~}
\newcommand{\secondaryinlineheading}[1]{\vspace{1ex}{\noindent}{\it #1.}~}

% Code style macros
\newcommand{\sql}[1]{{\normalfont{\texttt{#1}}}} % added \normalfont, so that SQL is not italicized inside definitions
\newcommand{\smsql}[1]{{\scriptsize{\sql{#1}}}}
\newcommand{\quot}[1]{\textquotesingle #1\textquotesingle}
\newcommand{\sqlquot}[1]{\sql{\quot{#1}}}

% Textual shortcuts
\newcommand{\sa}{{\it SchemaAnalyst}\xspace}
\newcommand{\SA}{\sa}
\newcommand{\SchemaAnalyst}{\sa}
\newcommand{\schemaanalyst}{\sa}
\newcommand{\sawebsite}{\url{http://www.schemaanalyst.org}}

\newcommand{\Original}{standard\xspace}
\newcommand{\Standard}{standard\xspace}
\newcommand{\VirtualMutationAnalysis}{Virtual Mutation Analysis\xspace}  % for captions and headings
\newcommand{\VMA}{\VirtualMutationAnalysis}                              % for captions and headings
\newcommand{\virtualmutationanalysis}{virtual mutation analysis\xspace}  % for prose
\newcommand{\vma}{\virtualmutationanalysis}                              % for prose

\newcommand{\NistWeather}{NistWeather\xspace}
\newcommand{\NW}{\NistWeather}

\newcommand{\DBMonster}{{\it DBMonster}\xspace}
\newcommand{\avm}{{\it AVM}\xspace}
\newcommand{\AVM}{\avm}
\newcommand{\theAVM}{the \avm}
\newcommand{\TheAVM}{The \avm}
\newcommand{\rand}{{\it Random$^{+}$}\xspace}
\newcommand{\random}{\rand}
\newcommand{\Random}{\rand}
\newcommand{\rnd}{\rand}

\newcommand{\postgres}{PostgreSQL\xspace}
\newcommand{\postgresql}{\postgres}
\newcommand{\Postgres}{\postgres}
\newcommand{\PostgreSQL}{\postgres}
\newcommand{\hypersql}{HyperSQL\xspace}
\newcommand{\hsqldb}{\hypersql}
\newcommand{\HSQLDB}{\hypersql}
\newcommand{\HyperSQL}{\hypersql}
\newcommand{\postgreshypersql}{\postgres/\hypersql}
\newcommand{\sqlite}{SQLite\xspace}
\newcommand{\SQLite}{\sqlite}

\newcommand{\CHECK}{\sql{CHECK}\xspace}
\newcommand{\CHECKs}{\sql{CHECK}s\xspace}
\newcommand{\CC}{\CHECK constraint\xspace}
\newcommand{\CCs}{\CHECK constraints\xspace}

\newcommand{\fk}{foreign key\xspace}
\newcommand{\fks}{foreign keys\xspace}
\newcommand{\FK}{\sql{FOREIGN KEY}\xspace}
\newcommand{\FKs}{\sql{FOREIGN KEY}s\xspace}
\newcommand{\FKC}{\FK constraint\xspace}
\newcommand{\FKCs}{\FK constraints\xspace}

\newcommand{\NOTNULL}{\sql{NOT NULL}\xspace}
\newcommand{\NOTNULLs}{\sql{NOT NULL}s\xspace}
\newcommand{\NNC}{\sql{NOT NULL} constraint\xspace}
\newcommand{\NNCs}{\sql{NOT NULL} constraints\xspace}

\newcommand{\pk}{primary key\xspace}
\newcommand{\pks}{primary keys\xspace}
\newcommand{\PK}{\sql{PRIMARY KEY}\xspace}
\newcommand{\PKs}{\sql{PRIMARY KEY}s\xspace}
\newcommand{\PKC}{\PK constraint\xspace}
\newcommand{\PKCs}{\PK constraints\xspace}

\newcommand{\UNIQUE}{\sql{UNIQUE}\xspace}
\newcommand{\UC}{\UNIQUE constraint\xspace}
\newcommand{\UNIQUEs}{\sql{UNIQUE}s\xspace}
\newcommand{\UCs}{\UNIQUE constraints\xspace}

\newcommand{\INSERT}{\sql{INSERT}\xspace}
\newcommand{\INSERTs}{\sql{INSERT}s\xspace}
\newcommand{\NULL}{\sql{NULL}\xspace}
\newcommand{\NULLs}{\sql{NULL}s\xspace}

\newcommand{\icp}{integrity constraint predicate\xspace}
\newcommand{\icps}{integrity constraint predicates\xspace}
\newcommand{\nc}{null condition\xspace}
\newcommand{\cc}{constraint condition\xspace}
\newcommand{\ccs}{constraint conditions\xspace}

\newcommand{\mannwhitney}{Mann--Whitney {\it U} test\xspace}
\newcommand{\wilcoxon}{Wilcoxon rank-sum test\xspace}
\newcommand{\pvalue}{{\it p}-value\xspace}
\newcommand{\bfpvalue}{\textbf{\textit{p}-value}\xspace}
\newcommand{\atwelve}{\^{A}\textsubscript{12}\xspace}
\newcommand{\bfatwelve}{\textbf{\textit{\^{A}}}$\mathbf{_{12}}$\xspace}
\newcommand{\taub}{{\large$\tau_{b}$}}

\newcommand{\etal}{et al.\xspace}

% Floats
\newcommand{\Figure}[1]{Figure~\ref{#1}\xspace}
\newcommand{\Section}[1]{Section~\ref{#1}\xspace}
\newcommand{\Table}[1]{Table~\ref{#1}\xspace}

%% To help ensure consistent float scaling
\def \algorithmfigurescalefactor {0.7}
\def \tablescalefactor {0.85}

%% Rotated table column type
\newcolumntype{R}[2]{%
    >{\adjustbox{minipage=6em,angle=#1,lap=\width-(#2)}\bgroup}%
    l%
    <{\egroup}%
}
\newcommand*\rot{\multicolumn{1}{R{60}{1.5em}}}

%% Highlight command for coloured highlight in listings
\newcommand{\reducedstrut}{\vrule width 0pt height \ht\strutbox depth \dp\strutbox\relax}
\newcommand{\highlight}[1]{%
  \begingroup
  \setlength{\fboxsep}{0pt}%
  \colorbox{red!20}{\reducedstrut#1\/}%
  \endgroup
}

% Equations
\newcommand{\eqnull}{= \relnull}
\newcommand{\existsinrel}[2]{\exists \inrel{#1}{#2}}
\newcommand{\forallinrel}[2]{\forall \inrel{#1}{#2}}
\newcommand{\inrel}[2]{#1 \in \: #2}
\newcommand{\neqnull}{\neq \relnull}
\newcommand{\relnull}{\bot}
\newcommand{\select}[2]{#1 (#2)}
\newcommand{\selecteqnull}[2]{\select{#1}{#2} \eqnull}
\newcommand{\selectneqnull}[2]{\select{#1}{#2} \neqnull}
\newcommand{\selectneq}[3]{\select{#1}{#3} \: \neq \: \select{#2}{#3}}
\newcommand{\selecteq}[3]{\select{#1}{#3} \: = \: \select{#2}{#3}}
\newcommand{\selecteqexp}[4]{\select{#1}{#3} \: = \: \select{#2}{#4}}
\newcommand{\smallwedge}{\: \wedge \:}
\newcommand{\smallvee}{\: \vee \:}

\let\oldemptyset\emptyset
\let\emptyset\varnothing

% Algorithms in figures
\newcommand{\tinybullet}{$\vcenter{\hbox{\tiny$\bullet$}} \;$}
\newcommand{\atab}{\hspace{3ex}}
\newcommand{\wheretab}{\hspace{2ex}}
\newcommand{\exptab}{\hspace{2.5ex}}
\newcommand{\smallvspace}{\vspace{0.5ex}}
\newcommand{\closeblock}{\vspace{1ex}}
\newcommand{\closemajorblock}{\vspace{2ex}}

% Comments
\newcommand{\rem}[1]{}

% RQs
\newcommand{\conclusion}[2]{\vspace{1mm} \noindent {\bf Conclusion for #1:} #2}
